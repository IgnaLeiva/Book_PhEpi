% Options for packages loaded elsewhere
\PassOptionsToPackage{unicode}{hyperref}
\PassOptionsToPackage{hyphens}{url}
%
\documentclass[
  letterpaper,
  paper=6in:9in,
  pagesize=pdftex,
  headinclude=on,
  footinclude=on,
  12pt]{scrbook}

\usepackage{amsmath,amssymb}
\usepackage{iftex}
\ifPDFTeX
  \usepackage[T1]{fontenc}
  \usepackage[utf8]{inputenc}
  \usepackage{textcomp} % provide euro and other symbols
\else % if luatex or xetex
  \usepackage{unicode-math}
  \defaultfontfeatures{Scale=MatchLowercase}
  \defaultfontfeatures[\rmfamily]{Ligatures=TeX,Scale=1}
\fi
\usepackage{lmodern}
\ifPDFTeX\else  
    % xetex/luatex font selection
\fi
% Use upquote if available, for straight quotes in verbatim environments
\IfFileExists{upquote.sty}{\usepackage{upquote}}{}
\IfFileExists{microtype.sty}{% use microtype if available
  \usepackage[]{microtype}
  \UseMicrotypeSet[protrusion]{basicmath} % disable protrusion for tt fonts
}{}
\makeatletter
\@ifundefined{KOMAClassName}{% if non-KOMA class
  \IfFileExists{parskip.sty}{%
    \usepackage{parskip}
  }{% else
    \setlength{\parindent}{0pt}
    \setlength{\parskip}{6pt plus 2pt minus 1pt}}
}{% if KOMA class
  \KOMAoptions{parskip=half}}
\makeatother
\usepackage{xcolor}
\setlength{\emergencystretch}{3em} % prevent overfull lines
\setcounter{secnumdepth}{5}
% Make \paragraph and \subparagraph free-standing
\makeatletter
\ifx\paragraph\undefined\else
  \let\oldparagraph\paragraph
  \renewcommand{\paragraph}{
    \@ifstar
      \xxxParagraphStar
      \xxxParagraphNoStar
  }
  \newcommand{\xxxParagraphStar}[1]{\oldparagraph*{#1}\mbox{}}
  \newcommand{\xxxParagraphNoStar}[1]{\oldparagraph{#1}\mbox{}}
\fi
\ifx\subparagraph\undefined\else
  \let\oldsubparagraph\subparagraph
  \renewcommand{\subparagraph}{
    \@ifstar
      \xxxSubParagraphStar
      \xxxSubParagraphNoStar
  }
  \newcommand{\xxxSubParagraphStar}[1]{\oldsubparagraph*{#1}\mbox{}}
  \newcommand{\xxxSubParagraphNoStar}[1]{\oldsubparagraph{#1}\mbox{}}
\fi
\makeatother

\usepackage{color}
\usepackage{fancyvrb}
\newcommand{\VerbBar}{|}
\newcommand{\VERB}{\Verb[commandchars=\\\{\}]}
\DefineVerbatimEnvironment{Highlighting}{Verbatim}{commandchars=\\\{\}}
% Add ',fontsize=\small' for more characters per line
\usepackage{framed}
\definecolor{shadecolor}{RGB}{241,243,245}
\newenvironment{Shaded}{\begin{snugshade}}{\end{snugshade}}
\newcommand{\AlertTok}[1]{\textcolor[rgb]{0.68,0.00,0.00}{#1}}
\newcommand{\AnnotationTok}[1]{\textcolor[rgb]{0.37,0.37,0.37}{#1}}
\newcommand{\AttributeTok}[1]{\textcolor[rgb]{0.40,0.45,0.13}{#1}}
\newcommand{\BaseNTok}[1]{\textcolor[rgb]{0.68,0.00,0.00}{#1}}
\newcommand{\BuiltInTok}[1]{\textcolor[rgb]{0.00,0.23,0.31}{#1}}
\newcommand{\CharTok}[1]{\textcolor[rgb]{0.13,0.47,0.30}{#1}}
\newcommand{\CommentTok}[1]{\textcolor[rgb]{0.37,0.37,0.37}{#1}}
\newcommand{\CommentVarTok}[1]{\textcolor[rgb]{0.37,0.37,0.37}{\textit{#1}}}
\newcommand{\ConstantTok}[1]{\textcolor[rgb]{0.56,0.35,0.01}{#1}}
\newcommand{\ControlFlowTok}[1]{\textcolor[rgb]{0.00,0.23,0.31}{\textbf{#1}}}
\newcommand{\DataTypeTok}[1]{\textcolor[rgb]{0.68,0.00,0.00}{#1}}
\newcommand{\DecValTok}[1]{\textcolor[rgb]{0.68,0.00,0.00}{#1}}
\newcommand{\DocumentationTok}[1]{\textcolor[rgb]{0.37,0.37,0.37}{\textit{#1}}}
\newcommand{\ErrorTok}[1]{\textcolor[rgb]{0.68,0.00,0.00}{#1}}
\newcommand{\ExtensionTok}[1]{\textcolor[rgb]{0.00,0.23,0.31}{#1}}
\newcommand{\FloatTok}[1]{\textcolor[rgb]{0.68,0.00,0.00}{#1}}
\newcommand{\FunctionTok}[1]{\textcolor[rgb]{0.28,0.35,0.67}{#1}}
\newcommand{\ImportTok}[1]{\textcolor[rgb]{0.00,0.46,0.62}{#1}}
\newcommand{\InformationTok}[1]{\textcolor[rgb]{0.37,0.37,0.37}{#1}}
\newcommand{\KeywordTok}[1]{\textcolor[rgb]{0.00,0.23,0.31}{\textbf{#1}}}
\newcommand{\NormalTok}[1]{\textcolor[rgb]{0.00,0.23,0.31}{#1}}
\newcommand{\OperatorTok}[1]{\textcolor[rgb]{0.37,0.37,0.37}{#1}}
\newcommand{\OtherTok}[1]{\textcolor[rgb]{0.00,0.23,0.31}{#1}}
\newcommand{\PreprocessorTok}[1]{\textcolor[rgb]{0.68,0.00,0.00}{#1}}
\newcommand{\RegionMarkerTok}[1]{\textcolor[rgb]{0.00,0.23,0.31}{#1}}
\newcommand{\SpecialCharTok}[1]{\textcolor[rgb]{0.37,0.37,0.37}{#1}}
\newcommand{\SpecialStringTok}[1]{\textcolor[rgb]{0.13,0.47,0.30}{#1}}
\newcommand{\StringTok}[1]{\textcolor[rgb]{0.13,0.47,0.30}{#1}}
\newcommand{\VariableTok}[1]{\textcolor[rgb]{0.07,0.07,0.07}{#1}}
\newcommand{\VerbatimStringTok}[1]{\textcolor[rgb]{0.13,0.47,0.30}{#1}}
\newcommand{\WarningTok}[1]{\textcolor[rgb]{0.37,0.37,0.37}{\textit{#1}}}

\providecommand{\tightlist}{%
  \setlength{\itemsep}{0pt}\setlength{\parskip}{0pt}}\usepackage{longtable,booktabs,array}
\usepackage{calc} % for calculating minipage widths
% Correct order of tables after \paragraph or \subparagraph
\usepackage{etoolbox}
\makeatletter
\patchcmd\longtable{\par}{\if@noskipsec\mbox{}\fi\par}{}{}
\makeatother
% Allow footnotes in longtable head/foot
\IfFileExists{footnotehyper.sty}{\usepackage{footnotehyper}}{\usepackage{footnote}}
\makesavenoteenv{longtable}
\usepackage{graphicx}
\makeatletter
\def\maxwidth{\ifdim\Gin@nat@width>\linewidth\linewidth\else\Gin@nat@width\fi}
\def\maxheight{\ifdim\Gin@nat@height>\textheight\textheight\else\Gin@nat@height\fi}
\makeatother
% Scale images if necessary, so that they will not overflow the page
% margins by default, and it is still possible to overwrite the defaults
% using explicit options in \includegraphics[width, height, ...]{}
\setkeys{Gin}{width=\maxwidth,height=\maxheight,keepaspectratio}
% Set default figure placement to htbp
\makeatletter
\def\fps@figure{htbp}
\makeatother

\usepackage{fvextra}
\DefineVerbatimEnvironment{Highlighting}{Verbatim}{breaklines,commandchars=\\\{\}}
\areaset[0.50in]{4.5in}{8in}
\makeatletter
\@ifpackageloaded{bookmark}{}{\usepackage{bookmark}}
\makeatother
\makeatletter
\@ifpackageloaded{caption}{}{\usepackage{caption}}
\AtBeginDocument{%
\ifdefined\contentsname
  \renewcommand*\contentsname{Table of contents}
\else
  \newcommand\contentsname{Table of contents}
\fi
\ifdefined\listfigurename
  \renewcommand*\listfigurename{List of Figures}
\else
  \newcommand\listfigurename{List of Figures}
\fi
\ifdefined\listtablename
  \renewcommand*\listtablename{List of Tables}
\else
  \newcommand\listtablename{List of Tables}
\fi
\ifdefined\figurename
  \renewcommand*\figurename{Figure}
\else
  \newcommand\figurename{Figure}
\fi
\ifdefined\tablename
  \renewcommand*\tablename{Table}
\else
  \newcommand\tablename{Table}
\fi
}
\@ifpackageloaded{float}{}{\usepackage{float}}
\floatstyle{ruled}
\@ifundefined{c@chapter}{\newfloat{codelisting}{h}{lop}}{\newfloat{codelisting}{h}{lop}[chapter]}
\floatname{codelisting}{Listing}
\newcommand*\listoflistings{\listof{codelisting}{List of Listings}}
\makeatother
\makeatletter
\makeatother
\makeatletter
\@ifpackageloaded{caption}{}{\usepackage{caption}}
\@ifpackageloaded{subcaption}{}{\usepackage{subcaption}}
\makeatother

\ifLuaTeX
  \usepackage{selnolig}  % disable illegal ligatures
\fi
\usepackage{bookmark}

\IfFileExists{xurl.sty}{\usepackage{xurl}}{} % add URL line breaks if available
\urlstyle{same} % disable monospaced font for URLs
\hypersetup{
  pdftitle={Pharmacoepidemiology course},
  pdfauthor={PhEpi Team},
  hidelinks,
  pdfcreator={LaTeX via pandoc}}


\title{Pharmacoepidemiology course}
\usepackage{etoolbox}
\makeatletter
\providecommand{\subtitle}[1]{% add subtitle to \maketitle
  \apptocmd{\@title}{\par {\large #1 \par}}{}{}
}
\makeatother
\subtitle{A course for pharmacist students}
\author{PhEpi Team}
\date{2023-10-19}

\begin{document}
\frontmatter
\maketitle

\RecustomVerbatimEnvironment{verbatim}{Verbatim}{
   showspaces = false,
   showtabs = false,
   breaksymbolleft={},
   breaklines
   % Note: setting commandchars=\\\{\} here will cause an error 
}  

\renewcommand*\contentsname{Table of contents}
{
\setcounter{tocdepth}{2}
\tableofcontents
}

\mainmatter
\bookmarksetup{startatroot}

\chapter*{Welcome to
pharmacoepidemiology}\label{welcome-to-pharmacoepidemiology}
\addcontentsline{toc}{chapter}{Welcome to pharmacoepidemiology}

\markboth{Welcome to pharmacoepidemiology}{Welcome to
pharmacoepidemiology}

These notes provide the core material for the pharmacoepidemiology
(PhEpi) course for pharmacist.

The curse provides an introduction to the key PhEpi concepts and
methods. Topics covered include study design, sources of bias and study
evaluation. It also make the diference between description, prediction
and causal inference. These topics provide the framework needed for
evaluating the existing literature. The module places a focus on
learning through practical examples and incorporates directed learning,
lectures, group discussion, and a final journal club.

This course aims are to introduce:

\begin{itemize}
\tightlist
\item
  the motivation and critical thinking towards solving a question in
  PhEpi studies through interrogation of data and drawing conclusions
  from evidence;
\item
  the principles of probability, regression modelling and statistical
  inference within frequentist and Bayesian frameworks.
\end{itemize}

\textbf{Course content}

Chapter~\ref{sec-measure-in-epi}

\bookmarksetup{startatroot}

\chapter{Measures in Epidemiology}\label{sec-measure-in-epi}

This lecture covers essential concepts for understanding how diseases
and other health outcomes are measured. We explore fundamental topics,
including prevalence and incidence, and discuss key measures such as
those that assess association between exposures and outcomes, as well as
measures that evaluate the effectiveness of interventions. While
measures of association help identify potential links, it's important to
distinguish these from causality, which requires further evidence to
establish a direct cause-and-effect relationship.

\section{Measures of frequency}\label{measures-of-frequency}

These tools are essential for describing and comparing populations,
tracking changes over time, and identifying potential associations
between exposures and health outcomes. Commonly referred to as Measures
of Morbidity, they provide crucial insights into the burden of disease
in a population and help guide public health decisions by quantifying
how often health events, such as illnesses or conditions, occur.

\subsection{Prevalence}\label{prevalence}

dddd

\subsection{Incidence}\label{incidence}

The question that remains unanswered though, is why use the Nix package
manager to install all this software instead of using the usual ways of
first installing R, and then using \texttt{install.packages()} to
install any required packages?

\(\text{Incidence rate per 1,000} =\)

\(\frac{
\begin{aligned}
    \text{Number of new cases of a disease} \\
    \text{occurring in a population during a} \\
    \text{specified period of time}
\end{aligned}
}{
\begin{aligned}
    \text{Total person-time (sum of the} \\
    \text{time periods of observation of each} \\
    \text{person who has been observed for all or} \\
    \text{part of the entire time period)}
\end{aligned}
} \times 1,000\)

\textbf{Key Differences Between Cumulative Incidence and Incidence Rate}

\begin{longtable}[]{@{}
  >{\raggedright\arraybackslash}p{(\columnwidth - 4\tabcolsep) * \real{0.1748}}
  >{\raggedright\arraybackslash}p{(\columnwidth - 4\tabcolsep) * \real{0.4336}}
  >{\raggedright\arraybackslash}p{(\columnwidth - 4\tabcolsep) * \real{0.3916}}@{}}
\toprule\noalign{}
\begin{minipage}[b]{\linewidth}\raggedright
\textbf{Aspect}
\end{minipage} & \begin{minipage}[b]{\linewidth}\raggedright
\textbf{Cumulative Incidence}
\end{minipage} & \begin{minipage}[b]{\linewidth}\raggedright
\textbf{Incidence Rate}
\end{minipage} \\
\midrule\noalign{}
\endhead
\bottomrule\noalign{}
\endlastfoot
\textbf{Measure} & Proportion or risk & Rate \\
\textbf{Formula} &
(\frac{\text{New cases}}{\text{Total population at risk}}) &
(\frac{\text{New cases}}{\text{Total person-time at risk}}) \\
\textbf{Unit} & No units (proportion) & Units of time (e.g., per
person-year) \\
\textbf{Assumptions} & Assumes all individuals are followed for the same
period & Accounts for varying follow-up times \\
\textbf{Interpretation} & Probability of developing disease over a
period & Speed at which new cases occur \\
\textbf{Best for} & Short-term studies with stable populations &
Long-term studies with varying follow-up times \\
\textbf{Population at risk} & Entire population at the start of the
period & Varies as it accounts for person-time \\
\end{longtable}

\section{Ensuring reproducibility with
Nix}\label{ensuring-reproducibility-with-nix}

The \texttt{nixpkgs} mono-repository is ``just'' a Github repsitory
which you can find here: \url{https://github.com/NixOS/nixpkgs}. This
repository contains Nix expressions to build and install more than
80'000 packages and you can search for installable Nix packages
\href{https://search.nixos.org/packages}{here}.

Because \texttt{nixpkgs} is a Github repository, it is possible to use a
specific commit hash to install the packages as they were at a specific
point in time. For example, if you use this commit, 7c9cc5a6e, you'll
get the very latest packages as of the 19th of October 2023, but if you
used this one instead: 976fa3369, you'll get packages from the 19th of
August 2023.

You can declare which revision of \texttt{nixpkgs} to use at the top of
a \texttt{default.nix} file. Here is what such a file looks like:

\begin{verbatim}
let
 pkgs = import (fetchTarball "https://github.com/NixOS/nixpkgs/archive/976fa3369d722e76f37c77493d99829540d43845.tar.gz") {};
 rpkgs = builtins.attrValues {
  inherit (pkgs.rPackages) tidymodels vetiver targets xgboost;
};
 system_packages = builtins.attrValues {
  inherit (pkgs) R;
};
in
 pkgs.mkShell {
  buildInputs = [  rpkgs system_packages  ];
 }
\end{verbatim}

As you can see, we import a specific revision of the \texttt{nixpkgs}
Github repository to ensure that we always get the same packages in our
environment.

If you're unfamiliar with Nix, this file can be quite scary. But don't
worry, with my co-author
\href{https://github.com/philipp-baumann}{Philipp Baumann} we developed
an R package called \texttt{\{rix\}} which generate this
\texttt{default.nix} files for you.

\section{The R \{rix\} package}\label{the-r-rix-package}

\texttt{\{rix\}} is an R package that makes it very easy to generate
very complex \texttt{default.nix} files. These files can in turn be used
by the Nix package manager to build project-specific environments. The
book's Github repository contains a file called \texttt{define\_env.R}
with the following content:

\begin{Shaded}
\begin{Highlighting}[]
\FunctionTok{library}\NormalTok{(rix)}

\FunctionTok{rix}\NormalTok{(}\AttributeTok{r\_ver =} \StringTok{"4.3.1"}\NormalTok{,}
    \AttributeTok{r\_pkgs =} \FunctionTok{c}\NormalTok{(}\StringTok{"quarto"}\NormalTok{),}
    \AttributeTok{system\_pkgs =} \StringTok{"quarto"}\NormalTok{,}
    \AttributeTok{tex\_pkgs =} \FunctionTok{c}\NormalTok{(}
      \StringTok{"amsmath"}\NormalTok{,}
      \StringTok{"framed"}\NormalTok{,}
      \StringTok{"fvextra"}\NormalTok{,}
      \StringTok{"environ"}\NormalTok{,}
      \StringTok{"fontawesome5"}\NormalTok{,}
      \StringTok{"orcidlink"}\NormalTok{,}
      \StringTok{"pdfcol"}\NormalTok{,}
      \StringTok{"tcolorbox"}\NormalTok{,}
      \StringTok{"tikzfill"}
\NormalTok{    ),}
    \AttributeTok{ide =} \StringTok{"other"}\NormalTok{,}
    \AttributeTok{shell\_hook =} \StringTok{""}\NormalTok{,}
    \AttributeTok{project\_path =} \StringTok{"."}\NormalTok{,}
    \AttributeTok{overwrite =} \ConstantTok{TRUE}\NormalTok{,}
    \AttributeTok{print =} \ConstantTok{TRUE}\NormalTok{)}
\end{Highlighting}
\end{Shaded}

\texttt{\{rix\}} ships the \texttt{rix()} function which takes several
arguments. These arguments allow you to specify an R version, a list of
R packages, a list of system packages, TeXLive packages and other
options that allow you to specify your requirements. Running this code
generates this \texttt{default.nix} file:

\begin{verbatim}
# This file was generated by the {rix} R package v0.4.1 on 2023-10-19
# with following call:
# >rix(r_ver = "976fa3369d722e76f37c77493d99829540d43845",
#  > r_pkgs = c("quarto"),
#  > system_pkgs = "quarto",
#  > tex_pkgs = c("amsmath",
#  > "framed",
#  > "fvextra",
#  > "environ",
#  > "fontawesome5",
#  > "orcidlink",
#  > "pdfcol",
#  > "tcolorbox",
#  > "tikzfill"),
#  > ide = "other",
#  > project_path = ".",
#  > overwrite = TRUE,
#  > print = TRUE,
#  > shell_hook = "")
# It uses nixpkgs' revision 976fa3369d722e76f37c77493d99829540d43845 for reproducibility purposes
# which will install R version 4.3.1
# Report any issues to https://github.com/b-rodrigues/rix
let
 pkgs = import (fetchTarball "https://github.com/NixOS/nixpkgs/archive/976fa3369d722e76f37c77493d99829540d43845.tar.gz") {};
 rpkgs = builtins.attrValues {
  inherit (pkgs.rPackages) quarto;
};
  tex = (pkgs.texlive.combine {
  inherit (pkgs.texlive) scheme-small amsmath framed fvextra environ fontawesome5 orcidlink pdfcol tcolorbox tikzfill;
});
 system_packages = builtins.attrValues {
  inherit (pkgs) R glibcLocalesUtf8 quarto;
};
  in
  pkgs.mkShell {
    LOCALE_ARCHIVE = if pkgs.system == "x86_64-linux" then  "${pkgs.glibcLocalesUtf8}/lib/locale/locale-archive" else "";
    LANG = "en_US.UTF-8";
    LC_ALL = "en_US.UTF-8";
    LC_TIME = "en_US.UTF-8";
    LC_MONETARY = "en_US.UTF-8";
    LC_PAPER = "en_US.UTF-8";
    LC_MEASUREMENT = "en_US.UTF-8";

    buildInputs = [  rpkgs tex system_packages  ];
  }
\end{verbatim}

You can now use this file to work on your book locally by first building
the environment and then use it. To know more about using
\texttt{default.nix} files on a day-to-day basis, read
\href{https://b-rodrigues.github.io/rix/articles/building-reproducible-development-environments-with-rix.html}{this
vignette}.

In the next chapter, I'm going to explain how this book gets built on
Github Actions.

\bookmarksetup{startatroot}

\chapter{Building on Github Actions with
Nix}\label{building-on-github-actions-with-nix}

\section{Setup}\label{setup}

Just like when building using the usual approches, you first need to
build the book locally, on your computer, once. For this, after having
generated the \texttt{default.nix} file, you can build the environment
using \texttt{nix-build}, and then drop in a shell with
\texttt{nix-shell} (if this previous sentence is confusing, make sure
you read the vignette linked at the end of the previous chapter).

Once in that shell, run \texttt{quarto\ publish\ gh-pages}. This will
render the book, and make sure that everything gets setup properly. If
the book does not render, this could mean that you're missing some
dependency. Make sure to specify all the requirements in the
\texttt{define\_env.R} script and that you re-generated the
\texttt{default.nix} file. If the \texttt{quarto\ publish\ gh-pages}
command succeeds, you're all set. Editing the book and pushing will
build the book on Github Actions.

\section{The Github Actions workflow
file}\label{the-github-actions-workflow-file}

Here is what the workflow file looks like:

\begin{verbatim}
name: Build book using Nix

on:
  push:
    branches:
      - main
      - master

jobs:
  build:
    runs-on: ubuntu-latest

    steps:
    - name: Checkout Code
      uses: actions/checkout@v3

    - name: Install Nix
      uses: DeterminateSystems/nix-installer-action@main
      with:
        logger: pretty
        log-directives: nix_installer=trace
        backtrace: full

    - name: Nix cache
      uses: DeterminateSystems/magic-nix-cache-action@main

    - name: Build development environment
      run: |
        nix-build

    - name: Publish to GitHub Pages (and render)
      uses: b-rodrigues/quarto-nix-actions/publish@main
      env:
        GITHUB_TOKEN: ${{ secrets.GITHUB_TOKEN }} 
\end{verbatim}

The first step \emph{Checkout code} makes the code available to the rest
of the steps. I then install Nix on this runner using the Determinate
Systems \texttt{nix-installer-action} and then I use another action from
Determinate Systems, the \texttt{magic-nix-cache-action}. This action
caches all the packages so that they don't need to get re-built each
time a change gets pushed, speeding up the process by a lot. The
development environment gets then built using \texttt{nix-build}.

Finally, an action I defined runs, \texttt{quarto-nix-actions/publish}.
This is a fork of the \texttt{quarto-actions/publish} action which you
can find
\href{https://github.com/quarto-dev/quarto-actions/blob/main/publish/action.yml}{here}.
My fork simply makes sure that the \texttt{quarto\ render} and
\texttt{quarto\ publish} commands run in the
\href{https://github.com/b-rodrigues/quarto-nix-actions/blob/f48f5a7813eb4978a2f557ff45bcc854526fb80b/publish/action.yml\#L58}{Nix
environment defined for the project}.

\bookmarksetup{startatroot}

\chapter{Conclusion}\label{conclusion}

So in conclusion, should you use this template? I think you should, even
if you're not that familiar with Nix. If you need to add packages,
simply rerun \texttt{define\_env.R} after having added the packages you
need. This will generate a new \texttt{default.nix} file that will
generate the right environment once you push this change. The advantage
of using Nix is that it will always work: the workflow file uses
\texttt{ubuntu-latest}, so the underlying operating system changes with
time, but because you're using a fixed revision of Nix, the same
versions of R and packages will get used, forever.

If you need more recent packages or a more recent version of R, simply
use a more recent \texttt{nixpkgs} revision. If, despite all these
advantages, you prefer using \texttt{\{renv\}}, you could check out my
\href{https://github.com/b-rodrigues/kdp_quarto}{other Github template}.
This template does exactly the same thing: it builds a website for your
book, an Epub for E-ink readers and a PDF, ready for Amazon's
self-publishing service. The difference is that the right version of R,
TeXLive and Quarto get installed using dedicated actions, and the R
packages get installed using \texttt{\{renv\}}. The underlying operating
system is \texttt{ubuntu-22.04} instead of \texttt{ubuntu-latest}. This
is to ensure that the underlying system dependencies stay stable, but it
also means that you will need to update this operating system once
version 22.04 of Ubuntu is deprecated (in 2027) which could cause the R
version and packages that you need not to be installable anymore. This
is a lot of moving pieces, and if one of them fails, nothing will work
anymore. You also notice this if you pay attention at the number of
lines of code of the workflow files of both repositories: if you compare
the
\href{https://github.com/b-rodrigues/kdp_quarto/blob/main/.github/workflows/build_book.yml}{Github
workflow file from the template that uses actions to install the right
software and \{renv\}} to the one from
\href{https://github.com/b-rodrigues/quarto_book_nix/blob/master/.github/workflows/build_book.yml}{this
template} you'll notice that the one from this template is much shorter
as well.

The only dependency is Nix itself, and Nix is not going anywhere, as
it's been around for 20 years. The Determinate System actions are
optional; so even if for some reason those fail in the future, they're
not needed. It's just that using them makes things easier.

If you use this template, or have any questions, please let me know by
opening an
\href{https://github.com/b-rodrigues/quarto_book_nix/issues}{issue}.

\bookmarksetup{startatroot}

\chapter{Odds Ratio and Risk Ratio
Calculator}\label{odds-ratio-and-risk-ratio-calculator}

\#```\{shinylive-r\} \#\textbar{} standalone: true \#\textbar{}
viewerHeight: 800 library(shinylive) library(shiny)

\bookmarksetup{startatroot}

\chapter{Define UI for the
application}\label{define-ui-for-the-application}

ui \textless- fluidPage( titlePanel(``\,``),

sidebarLayout( sidebarPanel( \# Dropdown to select OR or RR
selectInput(``measure'', ``Select Measure:'', choices = list(``Odds
Ratio (OR)'' = ``OR'', ``Risk Ratio (RR)'' = ``RR'')),

\begin{verbatim}
  # Input fields for A, B, C, and D
  numericInput("A_input", "A: Exposed cases", value = 20),
  numericInput("B_input", "B: Exposed non-cases", value = 80),
  numericInput("C_input", "C: Unexposed cases", value = 15),
  numericInput("D_input", "D: Unexposed non-cases", value = 85),
  
  # Dropdown to select CI level or enter custom
  numericInput("CI_level", "Confidence Level (%):", value = 95, min = 50, max = 99)
),

mainPanel(
  # Output text for the selected measure and custom CI
  textOutput("result")
)
\end{verbatim}

) )

\bookmarksetup{startatroot}

\chapter{Define server logic}\label{define-server-logic}

server \textless- function(input, output) \{

\# Reactive expression to calculate OR or RR and custom CI
output\$result \textless- renderText(\{

\begin{verbatim}
# Assign values from user inputs
A <- input$A_input
B <- input$B_input
C <- input$C_input
D <- input$D_input
CI_level <- input$CI_level / 100  # Convert CI level to decimal

# Calculate z-value based on CI level
z_value <- qnorm(1 - (1 - CI_level) / 2)

# Check whether to calculate OR or RR based on user input
if (input$measure == "OR") {
  # Calculate Odds Ratio (OR)
  OR <- (A * D) / (B * C)
  
  # Calculate custom CI for OR
  lower_CI <- exp(log(OR) - z_value * sqrt(1/A + 1/B + 1/C + 1/D))
  upper_CI <- exp(log(OR) + z_value * sqrt(1/A + 1/B + 1/C + 1/D))
  
  # Display OR result
  paste0("Odds Ratio (OR): ", round(OR, 2), 
         ", ", round(input$CI_level, 0), "% CI: ", round(lower_CI, 2), " - ", round(upper_CI, 2))
  
} else {
  # Calculate Risk Ratio (RR)
  risk_exposed <- A / (A + B)
  risk_unexposed <- C / (C + D)
  RR <- risk_exposed / risk_unexposed
  
  # Calculate custom CI for RR
  lower_CI <- exp(log(RR) - z_value * sqrt((1/A) - (1/(A+B)) + (1/C) - (1/(C+D))))
  upper_CI <- exp(log(RR) + z_value * sqrt((1/A) - (1/(A+B)) + (1/C) - (1/(C+D))))
  
  # Display RR result
  paste0("Risk Ratio (RR): ", round(RR, 2), 
         ", ", round(input$CI_level, 0), "% CI: ", round(lower_CI, 2), " - ", round(upper_CI, 2))
}
\end{verbatim}

\}) \}

\bookmarksetup{startatroot}

\chapter{Run the application}\label{run-the-application}

shinyApp(ui = ui, server = server) \#```


\backmatter


\end{document}
